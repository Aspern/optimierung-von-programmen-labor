\subsection{Sortieren durch direkte Auswahl}

Der Algorithmus \textit{Sortieren durch direkte Auswahl} (englisch Selectionsort) sortiert ein Array von Elementen in aufsteigender Reihenfolge. Dabei sucht der Algorithmus das kleinste Element im Array und vertuscht es mit der ersten Stelle im Array. Anschließend wird im Array ohne die erste Stelle erneut das Minimum gesucht und an die zweite Stelle getauscht. So entsteht ein sortierter Teil $S$ und ein unsortierter Teil $U$ im Array. Der Algorithmus fährt solange fort, bis in $U$ keine Elemente mehr vorhanden sind. Der Algorithmus liegt in der Komplexitätsklasse $O(n^2)$.

\begin{PseudoCode}
A := array with comparable elements;
for i = 0 to n - 1 do
	minimum = searchMin(A, i);
	switch(A[i], minimum);
end for
\end{PseudoCode}

\noindent
\textbf{Beispiel 3.2.1 - Ausführung Sortieren durch direkte Auswahl}

\noindent
Für die Ausführung des Algorithmus wird das Array $A$ und eine Zählvariable $i$ initialisiert. Zur Verdeutlichung werden wird $i$ blau und das aktuell gefundene Minimum rot markiert. Der bereits sortierte Teil des Arrays wird grün markiert
\begin{equation}
	A := \{7, 4, 17, 1, 6, 7\},\;i:=0
\end{equation}
\noindent
Der Algorithmus startet bei $A[0] = 7$, dazu wird im Bereich $A[0] - A[5]$ das Minimum $1$ gefunden.
\begin{equation}
	A = \{\textcolor{red}{7}, 4, 17, \textcolor{blue}{1}, 6, 7\},\;i = 0 \\
\end{equation}
\noindent
Anschließend wird das Minimum mit dem aktuellen Index vertauscht, da $7 > 1$.
\begin{equation}
	A = \{\textcolor{green}{1}, 4, 17, 7, 6, 7\},\;i = 0 \\
\end{equation}
\noindent
Nach dem vertauschen startet der nächste Schleifendurchlauf mit $A[1] = 4$.
\begin{equation}
	A = \{\textcolor{green}{1}, \textcolor{red}{4}, 17, 7, 6, 7\},\;i = 1 \\
\end{equation}
\noindent
In diesem Fall muss nicht getauscht werden da $4$ bereits das Minimum ist.
\begin{equation}
	A = \{\textcolor{green}{1}, \textcolor{green}{4}, 17, 7, 6, 7\},\;i = 1 \\
\end{equation}
\newpage
\noindent
Im nächsten Schritt wird A[2] = 17 mit dem gefundenen Minimum $6$ aus $A[2] - A[5]$ behandelt.
\begin{equation}
	A = \{\textcolor{green}{1}, \textcolor{green}{4}, \textcolor{red}{17}, 7, \textcolor{blue}{6}, 7\},\;i = 2 \\
\end{equation}
\noindent
Wieder wird vertauscht, da $17 > 6$.
\begin{equation}
	A = \{\textcolor{green}{1}, \textcolor{green}{4}, \textcolor{green}{6}, 7, 17, 7\},\;i = 2 \\
\end{equation}
\noindent
Für $A[3] = 7$ wird kein Minimum gefunden, da $7$ bereits das kleinste Element ist.
\begin{equation}
	A = \{\textcolor{green}{1}, \textcolor{green}{4}, \textcolor{green}{6},\textcolor{green}{7}, 17, 7\},\;i = 3 \\
\end{equation}
\noindent
Anschließend werden noch $A[4] = 17$ mit dem letzten Minimum $7$ aus $A[4] - A[5]$ betrachtet.
\begin{equation}
	A = \{\textcolor{green}{1}, \textcolor{green}{4}, \textcolor{green}{6},\textcolor{green}{7}, \textcolor{red}{17}, \textcolor{blue}{7}\},\;i = 4 \\
\end{equation}
\noindent
Die Elemente werden vertauscht, da $17 > 7$.
\begin{equation}
	A = \{\textcolor{green}{1}, \textcolor{green}{4}, \textcolor{green}{6},\textcolor{green}{7}, \textcolor{green}{7}, \textcolor{green}{17}\},\;i = 4 \\
\end{equation}
\noindent
Anschließend terminiert der Algorithmus, dass letzte Element muss nicht mehr geprüft werden, da es bereits im letzte Schritt behandelt wurde. 

\newpage



