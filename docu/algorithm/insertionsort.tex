\subsection{Sortieren durch Einfügen}
Der Algorithmus \textit{Sortieren durch Einfügen} (englisch Insertionsort) sortiert ein Array von Elementen in aufsteigender Reihenfolge. Dabei iteriert der Algorithmus in einer äußeren Schleife über das gesamte Array. Jedes Element wird anschließend in einer inneren Schleife mit jedem Vorgänger verglichen und vertauscht, solange dieser größer ist wie das aktuelle Element. Am Ende entsteht so ein vollständig sortiertes Array. Der Algorithmus liegt in der Komplexitätsklasse $O(n^2)$.

\begin{PseudoCode}
	A := array with comparable elements;
	for i = 1 to n do
		for j = i to j > 0 and A[j - 1] > A[j] do
			switch(A[j], A[j - 1]);
		end for
	end for
\end{PseudoCode}

\noindent
\textbf{Beispiel 3.3.1 - Ausführung Sortieren durch Einfügen}

\noindent
Für die Ausführung des Algorithmus wird das Array $A$ und zwei Zählvariablen $i$ und $j$ initialisiert. Zur Verdeutlichung wird $i$ blau, $j$ und $j-1$ rot und der bereits sortierte Teil des Arrays grün dargestellt.
\begin{equation*}
	A := \{7, 4, 17, 1, 6, 7\},\;i:=1,\;j:=0
\end{equation*}
\noindent
Der Algorithmus startet mit $i=1$ bei $A[1] = 4$ in der äußeren Schleife
\begin{equation*}
	A = \{7, \textcolor{blue}{4}, 17, 1, 6, 7\},\;i = 1,\;j=0 \\
\end{equation*}
\noindent
In der inneren Schleife wird $j = i = 1$ gesetzt und  $A[j] mit A[j - 1]$ verglichen.
\begin{equation*}
	A = \{\textcolor{red}{7}, \textcolor{red}{4}, 17, 1, 6, 7\},\;i = 1,\;j=1 \\
\end{equation*}
\noindent
Da $7 > 4$ werden die beiden Elemente vertauscht und $j$ anschließend dekrementiert.
\begin{equation*}
	A = \{\textcolor{red}{4}, 7, 17, 1, 6, 7\},\;i = 1,\;j=0 \\
\end{equation*}
\noindent
Da $j=0$ wird die innere Schleife abgebrochen und mit der äußeren fortgefahren.
\begin{equation*}
	A = \{4, 7, \textcolor{blue}{17}, 1, 6, 7\},\;i = 2,\;j=0 \\
\end{equation*}
\noindent
Anschließend wird wieder die innere Schleife mit $j=i=2$ ausgeführt.
\begin{equation*}
	A = \{4, \textcolor{red}{7}, \textcolor{red}{17}, 1, 6, 7\},\;i = 2,\;j=2 \\
\end{equation*}
\noindent
Da bereits $7 < 17$ muss nicht getauscht werden und die innere Schleife wird abgebrochen.
\begin{equation*}
	A = \{4, 7, 17, \textcolor{blue}{1}, 6, 7\},\;i = 3,\;j=2 \\
\end{equation*}
\noindent
Die innere Schleife wird jetzt mit $j=i=3$ ausgeführt. Da $17 > 1$ werden die Elemente $A[j]$ mit $A[j-1]$ vertauscht. Das wiederholt sich bis $j=0$, da außerdem $7 > 1$ und $4>1$. Anschließend terminiert die innere Schleife wegen $j=0$.
\begin{equation*}
	\begin{split}
		A = \{4, 7,  \textcolor{red}{17}, \textcolor{red}{1}, 6, 7\},\;i = 3,\;j=3 \\
		A = \{4, \textcolor{red}{7},  \textcolor{red}{1}, 17, 6, 7\},\;i = 3,\;j=2 \\
		A = \{\textcolor{red}{4}, \textcolor{red}{1},  7, 17, 6, 7\},\;i = 3,\;j=1 \\
		A = \{\textcolor{red}{1}, 4,  7, 17, 6, 7\},\;i = 3,\;j=0 \\
	\end{split}
\end{equation*}
\noindent
Die äußere Schleife wird jetzt mit $i=4$ fortgesetzt.
\begin{equation*}
	A = \{1,4,7,17,\textcolor{blue}{6},7\},\;i = 4,\;j=0 \\
\end{equation*}
\noindent
Die innere Schleife wird dann mit $j=i=4$ ausgeführt. Da $17 > 6$ werden die Elemente $A[j]$ mit $A[j-1]$ vertauscht. Das wiederholt sich bis $j=2$, da außerdem $7 > 6$. Anschließend terminiert die innere Schleife wegen $A[j-1] < A[j]$.
\begin{equation*}
	\begin{split}
		A = \{1, 4, 7, \textcolor{red}{17}, \textcolor{red}{6}, 7\},\;i = 4,\;j=4 \\
		A = \{1, 4, \textcolor{red}{7}, \textcolor{red}{6}, 17, 7\},\;i = 4,\;j=3 \\
		A = \{1, \textcolor{red}{4}, \textcolor{red}{6}, 7, 17, 7\},\;i = 4,\;j=2 \\
	\end{split}
\end{equation*}
\noindent
Die äußere Schleife wird mit dem letzten Element $i=5$ fortgesetzt.
\begin{equation*}
	A = \{1,4,6,7,17,\textcolor{blue}{7}\},\;i = 4,\;j=2 \\
\end{equation*}
\newpage
\noindent
Die innere Schleife startet bei $j=i=5$ und vergleicht $A[j-1]$ mit $A[j]$. Da $17 > 7$ werden die Elemente vertauscht, danach bricht die Schleife ab da $A[j-1] = A[j]$.
\begin{equation*}
	\begin{split}
		A = \{1,4,6,7,\textcolor{red}{17},\textcolor{red}{7}\},\;i = 4,\;j=3 \\
		A = \{1,4,6,\textcolor{red}{7},\textcolor{red}{7}, 17\},\;i = 4,\;j=3 \\
	\end{split}
\end{equation*}
\noindent
Anschließend Terminiert die äußere Schleife wegen $i = N$ und der Algorithmus ist fertig.
\begin{equation*}
	A = \{\textcolor{green}{1},\textcolor{green}{4},\textcolor{green}{6},\textcolor{green}{7},\textcolor{green}{7},\textcolor{green}{17}\},\;i = 4,\;j=2 \\
\end{equation*}
\newpage