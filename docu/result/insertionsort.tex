\subsection{Sortieren durch direktes Einfügen}

Die nachfolgenden Laufzeiten vergleichen zwei unterschiedliche Implementierungen des Algorithmus \textit{Sortieren durch Einfügen}. In der ersten Variante wird mit zwei Schleifen sortiert, in der zweiten Abwandlung wird die innere Schleife ausgerollt und ein Prefetch-Befehl verwendet.

\subsubsection{Variante 1: Normal}

\begin{center}
	\begin{longtable}{|R{5cm}|R{3cm}|R{3cm}|R{3cm}|}
		\hline
		
		% Tabelleninhalt
		\multirow{2}{5cm}{\centering \textbf{Feldgröße [n]}} & \multicolumn{3}{|c|}{ \textbf{Laufzeit [$ms$]}} \\\cline{2-4}
		& \multicolumn{1}{|c|}{\textbf{ASC}} & \multicolumn{1}{|c|}{\textbf{DESC}} &\multicolumn{1}{|c|}{\textbf{RAND}} \\
		\hhline{|=|=|=|=|}
		
		16384 & 0 & 282 & 139\\
		\hline
		32768 & 0 & 1113 & 554\\
		\hline
		65536 & 0 & 4525 & 2256\\
		\hline
		131072 & 0 & 18043 & 9026\\
		\hline
		262144 & 0 & 72410 & 36246\\
		\hline
		524288 & 0 & 290495 & 144620\\
		\hline
		1048576 & 1 & 1190247 & 587934\\
		\hline
		
		% Tabellenbezeichung
		\caption{Sortieren durch Einfügen (Normal).}
		\label{tab:insertionsort-v1}
	\end{longtable}
\end{center}

\subsubsection{Variante 2: Mit Prefetch}

\begin{center}
	\begin{longtable}{|R{5cm}|R{3cm}|R{3cm}|R{3cm}|}
		\hline
		
		% Tabelleninhalt
		\multirow{2}{5cm}{\centering \textbf{Feldgröße [n]}} & \multicolumn{3}{|c|}{ \textbf{Laufzeit [$ms$]}} \\\cline{2-4}
		& \multicolumn{1}{|c|}{\textbf{ASC}} & \multicolumn{1}{|c|}{\textbf{DESC}} &\multicolumn{1}{|c|}{\textbf{RAND}} \\
		\hhline{|=|=|=|=|}
		
		16384 & 19 & 162 & 91\\
		\hline
		32768 & 80 & 633 & 365\\
		\hline
		65536 & 429 & 2780 & 1596\\
		\hline
		131072 & 1907 & 11194 & 6578\\
		\hline
		262144 & 7791 & 44479 & 26199\\
		\hline
		524288 & 32901 & 179240 & 105709\\
		\hline
		1048576 & 198462 & 862873 & 509700\\
		\hline
		
		% Tabellenbezeichung
		\caption{Sortieren durch Einfügen mit Prefetch.}
		\label{tab:insertionsort-v2}
	\end{longtable}
\end{center}

\subsubsection{Interpretation}

Bei der normalen Variante bestätigt sich die Komplexität von $O(n)$ im besten Fall, sowie $O(n^2)$ im durchschnittlichen und schlechtesten Fall. Wie zu erwarten sind die Laufzeiten bei absteigend sortierten Folgen deutlich größer, da hier jedes Element bis nach vorne getauscht werden muss.

Die optimierte Variante verhält sich im durchschnittlichen und schlechtesten Fall gleich wie die Normale. Zusätzlich ist die Laufzeit hier durch den Prefetch-Befehl besser geworden. Im besten Fall gab es jedoch eine deutliche Verschlechterung, hier wächst die Komplexität plötzlich quadratisch, möglicherweise ausgelöst durch den an dieser Stelle unnötig ausgeführten Prefetch.
