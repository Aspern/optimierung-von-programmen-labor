\subsection{Minimumsuche}

Folgende Zeitmessungen zeigen die Laufzeiten der Minimumsuche mit drei verschiedenen Implementierungen: Normal, mit ausgerollten Schleifen und mit einem Prefetch-Befehl.

\subsubsection{Variante 1: Normal}

\begin{center}
	\begin{longtable}{|R{5cm}|R{3cm}|R{3cm}|R{3cm}|}
		\hline
		
		% Tabelleninhalt
		\multirow{2}{5cm}{\centering \textbf{Feldgröße [n]}} & \multicolumn{3}{|c|}{ \textbf{Laufzeit [$\mu s$]}} \\\cline{2-4}
		& \multicolumn{1}{|c|}{\textbf{ASC}} & \multicolumn{1}{|c|}{\textbf{DESC}} &\multicolumn{1}{|c|}{\textbf{RAND}} \\
		\hhline{|=|=|=|=|}
		
		16384 & 30 & 29 & 29\\
		\hline
		32768 & 54 & 61 & 58\\
		\hline
		65536 & 111 & 123 & 108\\
		\hline
		131072 & 233 & 232 & 213\\
		\hline
		262144 & 421 & 479 & 425\\
		\hline
		524288 & 880 & 886 & 903\\
		\hline
		1048576 & 1740 & 1737 & 1743\\
		\hline
		2097152 & 3764 & 3751 & 3566\\
		\hline
		4194304 & 6926 & 6951 & 7049\\
		\hline
		8388608 & 13348 & 13548 & 13558\\
		\hline
		16777216 & 27043 & 27975 & 28325\\
		\hline
		33554432 & 55425 & 53431 & 54039\\
		\hline
		67108864 & 109768 & 106431 & 108913\\
		\hline
		134217728 & 202659 & 206078 & 204265\\
		\hline
		268435456 & 409796 & 400175 & 444034\\
		\hline
		536870912 & 1077360 & 961671 & 955176\\
		\hline
		
		% Tabellenbezeichung
		\caption{Normale Minimumsuche.}
		\label{tab:minimumsearch-v1}
	\end{longtable}
\end{center}

\newpage

\subsubsection{Variante 2: Mit Schleifen Ausrollen}

\begin{center}
	\begin{longtable}{|R{5cm}|R{3cm}|R{3cm}|R{3cm}|}
		\hline
		
		% Tabelleninhalt
		\multirow{2}{5cm}{\centering \textbf{Feldgröße [n]}} & \multicolumn{3}{|c|}{ \textbf{Laufzeit [$\mu s$]}} \\\cline{2-4}
		& \multicolumn{1}{|c|}{\textbf{ASC}} & \multicolumn{1}{|c|}{\textbf{DESC}} &\multicolumn{1}{|c|}{\textbf{RAND}} \\
		\hhline{|=|=|=|=|}
		
		16384 & 31 & 30 & 28\\
		\hline
		32768 & 58 & 64 & 59\\
		\hline
		65536 & 106 & 108 & 109\\
		\hline
		131072 & 216 & 233 & 230\\
		\hline
		262144 & 426 & 441 & 443\\
		\hline
		524288 & 901 & 914 & 889\\
		\hline
		1048576 & 1746 & 1758 & 1773\\
		\hline
		2097152 & 3530 & 3518 & 3513\\
		\hline
		4194304 & 7045 & 7041 & 7035\\
		\hline
		8388608 & 13597 & 13635 & 14096\\
		\hline
		16777216 & 27933 & 28018 & 27096\\
		\hline
		33554432 & 53879 & 55001 & 52187\\
		\hline
		67108864 & 104367 & 101978 & 104117\\
		\hline
		134217728 & 216644 & 216945 & 215795\\
		\hline
		268435456 & 444691 & 460349 & 466158\\
		\hline
		536870912 & 840530 & 813142 & 899364\\
		\hline
		
		% Tabellenbezeichung
		\caption{Minimumsuche mit Schleifen ausrollen}
		\label{tab:minimumsearch-v2}
	\end{longtable}
\end{center}

\newpage

\subsubsection{Variante 3: Mit Prefetch}

\begin{center}
	\begin{longtable}{|R{5cm}|R{3cm}|R{3cm}|R{3cm}|}
		\hline
		
		% Tabelleninhalt
		\multirow{2}{5cm}{\centering \textbf{Feldgröße [n]}} & \multicolumn{3}{|c|}{ \textbf{Laufzeit [$\mu s$]}} \\\cline{2-4}
		& \multicolumn{1}{|c|}{\textbf{ASC}} & \multicolumn{1}{|c|}{\textbf{DESC}} &\multicolumn{1}{|c|}{\textbf{RAND}} \\
		\hhline{|=|=|=|=|}
		
		16384 & 26 & 27 & 26\\
		\hline
		32768 & 57 & 54 & 53\\
		\hline
		65536 & 98 & 99 & 97\\
		\hline
		131072 & 197 & 196 & 198\\
		\hline
		262144 & 395 & 397 & 395\\
		\hline
		524288 & 841 & 831 & 831\\
		\hline
		1048576 & 1617 & 1614 & 1617\\
		\hline
		2097152 & 3286 & 3280 & 3286\\
		\hline
		4194304 & 6532 & 6412 & 6376\\
		\hline
		8388608 & 12936 & 12353 & 12426\\
		\hline
		16777216 & 25763 & 26332 & 25283\\
		\hline
		33554432 & 50038 & 49740 & 52096\\
		\hline
		67108864 & 98898 & 106837 & 103934\\
		\hline
		134217728 & 203489 & 198460 & 199950\\
		\hline
		268435456 & 408708 & 381526 & 407902\\
		\hline
		536870912 & 834184 & 831809 & 1117660\\
		\hline
		
		% Tabellenbezeichung
		\caption{Minimumsuche mit Prefetch.}
		\label{tab:minimumsearch-v3}
	\end{longtable}
\end{center}

\subsubsection{Interpretation}

Bei genauerer Betrachtung der Tabellen \ref{tab:minimumsearch-v1}, \ref{tab:minimumsearch-v2} und \ref{tab:minimumsearch-v3} sind zunächst keine großen Unterschiede zu erkennen. Die Laufzeit wächst, wie erwartet, konstant mit der Feldgröße. Auch zwischen den unterschiedlichen Varianten der Array-Befüllung sind keine nennenswerten Auswirkungen auf die Laufzeit festzustellen, da in jedem Fall immer alle Elemente untersucht werden müssen.
Die Variante mit Schleifen ausrollen ist etwas schneller als die normale Minimumsuche, was zu erwarten war. Unerwartet sind die Laufzeiten der Prefetch-Minimumsuche im Vergleich zur Variante mit Schleifen ausrollen. Die Minimumsuche mit Prefetch scheint etwas langsamer zu sein, dies könnte ein Indiz dafür sein, dass die Schleifen nicht ausgerollt werden.