\subsection{Quicksort}

Nachfolgend werden die Laufzeiten von zwei Variante des \textit{Quicksort} Algorithmus verglichen. Einmal wird der Quicksort mit Three-Way-Partitioning, das zweite mal Hybrid implementiert.

\subsubsection{Variante 1: Three-Way-Partitioning}

\begin{center}
	\begin{longtable}{|p{5cm}|p{3cm}|p{3cm}|p{3cm}|}
		\hline
		
		% Tabelleninhalt
		\multirow{2}{5cm}{\centering \textbf{Feldgröße [n]}} & \multicolumn{3}{|c|}{ \textbf{Laufzeit [$ms$]}} \\\cline{2-4}
		& \multicolumn{1}{|c|}{\textbf{ASC}} & \multicolumn{1}{|c|}{\textbf{DESC}} &\multicolumn{1}{|c|}{\textbf{RAND}} \\
		\hhline{|=|=|=|=|}
		
		1048576 & 24 & 26 & 111\\
		\hline
		2097152 & 50 & 54 & 238\\
		\hline
		4194304 & 105 & 114 & 497\\
		\hline
		8388608 & 218 & 239 & 1064\\
		\hline
		16777216 & 458 & 498 & 2189\\
		\hline
		33554432 & 950 & 1021 & 4601\\
		\hline
		67108864 & 1977 & 2139 & 9435\\
		\hline
		134217728 & 4129 & 4420 & 20014\\
		\hline
		
		% Tabellenbezeichung
		\caption{Sortieren durch Einfügen mit Prefetch.}
		\label{tab:insertionsort-v2}
	\end{longtable}
\end{center}

\subsubsection{Variante 2: Hybrid}

\begin{center}
	\begin{longtable}{|p{5cm}|p{3cm}|p{3cm}|p{3cm}|}
		\hline
		
		% Tabelleninhalt
		\multirow{2}{5cm}{\centering \textbf{Feldgröße [n]}} & \multicolumn{3}{|c|}{ \textbf{Laufzeit [$ms$]}} \\\cline{2-4}
		& \multicolumn{1}{|c|}{\textbf{ASC}} & \multicolumn{1}{|c|}{\textbf{DESC}} &\multicolumn{1}{|c|}{\textbf{RAND}} \\
		\hhline{|=|=|=|=|}
		
		1048576 & 111 & 113 & 165\\
		\hline
		2097152 & 238 & 240 & 347\\
		\hline
		4194304 & 501 & 510 & 728\\
		\hline
		8388608 & 1067 & 1078 & 1552\\
		\hline
		16777216 & 2248 & 2278 & 3247\\
		\hline
		33554432 & 4722 & 4776 & 6675\\
		\hline
		67108864 & 9893 & 10055 & 14204\\
		\hline
		134217728 & 21060 & 21035 & 29604\\
		\hline
		
		% Tabellenbezeichung
		\caption{Sortieren durch Einfügen mit Prefetch.}
		\label{tab:insertionsort-v2}
	\end{longtable}
\end{center}

\subsubsection{Interpretation}